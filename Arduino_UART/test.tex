\documentclass[12pt]{article}
\usepackage{graphicx}
\usepackage[none]{hyphenat}
\usepackage[english]{babel}
\usepackage{caption}
\usepackage[parfill]{parskip}
\usepackage{hyperref}
\usepackage{import}
\usepackage{booktabs}
\usepackage{circuit}%%Importing Packing circiut.sty created for circuit
\def\inputGnumericTable{}
\usepackage{color}                                            
    \usepackage{array}                                            
    \usepackage{longtable}                                        
    \usepackage{calc}                                             
    \usepackage{multirow}                                         
    \usepackage{hhline}                                           
    \usepackage{ifthen}
\usepackage{array}
\usepackage{amsmath}  
\usepackage{parallel,enumitem}
\usepackage{listings}
\lstset{
language=tex,
frame=single,
breaklines=true
}
 
\begin{document}

\section{Flash Vaman-ESP using Arduino}

\renewcommand{\theequation}{\theenumi}
\renewcommand{\thefigure}{\theenumi}
\begin{enumerate}[label=\thesubsection.\arabic*.,ref=\thesubsection.\theenumi]
\numberwithin{equation}{enumi}
\numberwithin{figure}{enumi}
\numberwithin{table}{enumi}
\item Since the operating voltage of VAMAN board is 3.3 volts. Inorder to Flash code to VAMAN-ESP using the Arduino instead of USB-UART make connections as shown in Table \ref{tab:rpi-vaman-uart} and modify the circuit accordingly
			\begin{table}[!h]
		\input{tables/rpi-vaman-uart.tex}
		\caption{}
		\label{tab:rpi-vaman-uart}
	\end{table}
\item Add the Potential Divider circuit between VAMAN and ARDUINO boards TX and RX pins as shown in Figure \ref{fig:1}
\begin{figure}
\centering
\subimport{figs/}
{circuit.tex}
\caption{Circuit Connections}
\label{fig:1}
\end{figure}
\item Modify your platformio.ini file by adding the lines
\begin{lstlisting}
upload_protocol = esptool
upload_port = /dev/ttyACM0
upload_speed = 115200
\end{lstlisting}

\item After executing 
\begin{lstlisting}
pio run -t nobuild -t upload
\end{lstlisting}
while the dots and dashes are printed on the screen, disconnect the EN wire from GND.   Make sure that the Vaman board is not powering any device while flashing.  The Vaman-ESP should now flash.
\item After flashing, disconnect pin 0 on Vaman-ESP from GND. Power on Vaman and the appropriate LED will blink.
\end{enumerate}
\end{document}