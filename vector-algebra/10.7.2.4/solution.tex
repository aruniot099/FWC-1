\documentclass[12pt]{article}
\usepackage{graphicx}
%\documentclass[journal,12pt,twocolumn]{IEEEtran}
\usepackage[none]{hyphenat}
\usepackage{graphicx}
\usepackage{listings}
\usepackage[english]{babel}
\usepackage{graphicx}
\usepackage{caption} 
\usepackage{hyperref}
\usepackage{booktabs}
\def\inputGnumericTable{}
\usepackage{color}                                            %%
    \usepackage{array}                                            %%
    \usepackage{longtable}                                        %%
    \usepackage{calc}                                             %%
    \usepackage{multirow}                                         %%
    \usepackage{hhline}                                           %%
    \usepackage{ifthen}
\usepackage{array}
\usepackage{amsmath}   % for having text in math mode
\usepackage{listings}
\lstset{
language=tex,
frame=single, 
breaklines=true
}
  
%Following 2 lines were added to remove the blank page at the beginning
\usepackage{atbegshi}% http://ctan.org/pkg/atbegshi
\AtBeginDocument{\AtBeginShipoutNext{\AtBeginShipoutDiscard}}
%
%New macro definitions
\newcommand{\mydet}[1]{\ensuremath{\begin{vmatrix}#1\end{vmatrix}}}
\providecommand{\brak}[1]{\ensuremath{\left(#1\right)}}
\providecommand{\norm}[1]{\left\lVert#1\right\rVert}
\newcommand{\solution}{\noindent \textbf{Solution: }}
\newcommand{\myvec}[1]{\ensuremath{\begin{pmatrix}#1\end{pmatrix}}}
\let\vec\mathbf
\begin{document}
\begin{center}
\title{\textbf{Coordinate Geometry}}
\date{\vspace{-5ex}} %Not to print date automatically
\maketitle
\end{center}
\setcounter{page}{1}
\section*{10$^{th}$ Maths - Chapter 7}
This is Problem-4 from Exercise 7.2
\begin{enumerate}
\item Find the ratio in which the line segement joining the points $\myvec{-3 \\ 10}$ and $\myvec{6\\-8}$ is divided by $\myvec{-1\\6}$.\\
\solution \\The input parameters for this problem are available in Table \eqref{Table-1}
\begin{table}[ht!]
\input{table/table.tex}
\caption{}
\label{Table-1} 
\end{table}
Using section formula,
\begin{align}
         \vec{R}=\frac{\vec{Q}+n\vec{P}}{1+n}\label{eq:1}
\end{align}
Substituting the values of $\vec{P},\vec{Q}$ and $\vec{R}$ in \eqref{eq:1}
\begin{align}
         \myvec{-1\\6}=\frac{{\myvec{-3\\10}+n\myvec{6\\-8}}}{1+n}
\end{align}
\begin{align}
         \implies \myvec{-1\\6}=\frac{1}{1+n}\brak{{\myvec{-3\\10}+n\myvec{6\\-8}}} 
         \end{align}
\begin{align}
         \implies \myvec{-1\\6}=\frac{1}{1+n}\myvec{-3+6n\\10-8n} \label{eq:4}
\end{align}
Simplifying the \eqref{eq:4} yeilds,
\begin{align}
          -1 = \frac{-3+6n}{1+n}\\
         \implies n =\frac{2}{7}
\end{align}
and also,
\begin{align}
         6 =\frac{10-8n}{1+n}\\
          \implies n =\frac{2}{7}
\end{align}
Hence the ratio $n$ is $\dfrac{2}{7}$.
\begin{figure}[!h]
 \begin{center}
  \includegraphics[width=\columnwidth]{figs/Figure_1.png}
 \end{center}
\caption{}
\label{fig:Fig1}
\end{figure}
\end{enumerate}
\end{document}
