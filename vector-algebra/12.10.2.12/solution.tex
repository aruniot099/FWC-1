\documentclass[12pt]{article}
\usepackage{graphicx}
%\documentclass[journal,12pt,twocolumn]{IEEEtran}
\usepackage[none]{hyphenat}
\usepackage{graphicx}
\usepackage{listings}
\usepackage[english]{babel}
\usepackage{graphicx}
\usepackage{caption}
\usepackage[parfill]{parskip}
\usepackage{hyperref}
\usepackage{booktabs}
%\usepackage{setspace}\doublespacing\pagestyle{plain}
\def\inputGnumericTable{}
\usepackage{color}                                            %%
    \usepackage{array}                                            %%
    \usepackage{longtable}                                        %%
    \usepackage{calc}                                             %%
    \usepackage{multirow}                                         %%
    \usepackage{hhline}                                           %%
    \usepackage{ifthen}
\usepackage{array}
\usepackage{amsmath}   % for having text in math mode
\usepackage{parallel,enumitem}
\usepackage{listings}
\lstset{
language=tex,
frame=single,
breaklines=true
}
 
%Following 2 lines were added to remove the blank page at the beginning
\usepackage{atbegshi}% http://ctan.org/pkg/atbegshi
\AtBeginDocument{\AtBeginShipoutNext{\AtBeginShipoutDiscard}}
%
%New macro definitions
\newcommand{\mydet}[1]{\ensuremath{\begin{vmatrix}#1\end{vmatrix}}}
\providecommand{\brak}[1]{\ensuremath{\left(#1\right)}}
\providecommand{\norm}[1]{\left\lVert#1\right\rVert}
\newcommand{\solution}{\noindent \textbf{Solution: }}
\newcommand{\myvec}[1]{\ensuremath{\begin{pmatrix}#1\end{pmatrix}}}
\let\vec\mathbf
\begin{document}
\begin{center}
\enlargethispage{-4cm}
\title{\textbf{Straight Lines}}
\date{\vspace{-5ex}} %Not to print date automatically
\maketitle
\end{center}
\setcounter{page}{1}
\section*{12$^{th}$ Maths - Chapter 10}
This is Problem-12 from Exercise 10.2
\begin{enumerate}
\item Find the direction cosines of the vector $\hat{i} +2\hat{j}+3\hat{k}$.

\solution The direction cosines are the cosines of the angles formed by the given vector with the respective axes, let $\vec{A}$ be the given vector
\begin{align}
A =\hat{i} +2\hat{j}+3\hat{k}
\end{align}
The magnitude of the given vector is given by,
\begin{align}
\norm{\vec{A}}&=\sqrt{1^2+2^2+3^2}\\
\norm{\vec{A}} &=\sqrt{14}
\end{align}
The direction cosines of $\vec{A}$ can be expressed as
\begin{align}
a &=\frac{(i.A)}{\norm{\vec{A}}}\\
b &=\frac{(j.A)}{\norm{\vec{A}}}\\
c &=\frac{(k.A)}{\norm{\vec{A}}}
\end{align}
The dot product of the unit vectors in the direction of the $x,y$ and $z$ axes with vector are expressed as
\begin{align}
i.(\hat{i} +2\hat{j}+3\hat{k}) &=1\\
j.(\hat{i} +2\hat{j}+3\hat{k}) &=2\\
k.(\hat{i} +2\hat{j}+3\hat{k}) &=3
\end{align}
So the direction cosines of vector $\hat{i} +2\hat{j}+3\hat{k}$ are
\begin{align}
a &=\frac{(i.A)}{\norm{\vec{A}}}= \frac{1}{\sqrt{14}}\\
b &=\frac{(j.A)}{\norm{\vec{A}}}= \frac{2}{\sqrt{14}}\\
c &=\frac{(k.A)}{\norm{\vec{A}}}= \frac{3}{\sqrt{14}}
\end{align}



\end{enumerate}
\end{document}