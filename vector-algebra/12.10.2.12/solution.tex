\documentclass[12pt]{article}
\usepackage{graphicx}
%\documentclass[journal,12pt,twocolumn]{IEEEtran}
\usepackage[none]{hyphenat}
\usepackage{graphicx}
\usepackage{listings}
\usepackage[english]{babel}
\usepackage{graphicx}
\usepackage{caption}
\usepackage[parfill]{parskip}
\usepackage{hyperref}
\usepackage{booktabs}
%\usepackage{setspace}\doublespacing\pagestyle{plain}
\def\inputGnumericTable{}
\usepackage{color}                                            %%
    \usepackage{array}                                            %%
    \usepackage{longtable}                                        %%
    \usepackage{calc}                                             %%
    \usepackage{multirow}                                         %%
    \usepackage{hhline}                                           %%
    \usepackage{ifthen}
\usepackage{array}
\usepackage{amsmath}   % for having text in math mode
\usepackage{parallel,enumitem}
\usepackage{listings}
\lstset{
language=tex,
frame=single,
breaklines=true
}
 
%Following 2 lines were added to remove the blank page at the beginning
\usepackage{atbegshi}% http://ctan.org/pkg/atbegshi
\AtBeginDocument{\AtBeginShipoutNext{\AtBeginShipoutDiscard}}
%
%New macro definitions
\newcommand{\mydet}[1]{\ensuremath{\begin{vmatrix}#1\end{vmatrix}}}
\providecommand{\brak}[1]{\ensuremath{\left(#1\right)}}
\providecommand{\norm}[1]{\left\lVert#1\right\rVert}
\newcommand{\solution}{\noindent \textbf{Solution: }}
\newcommand{\myvec}[1]{\ensuremath{\begin{pmatrix}#1\end{pmatrix}}}
\let\vec\mathbf
\begin{document}
\begin{center}
\enlargethispage{-4cm}
\title{\textbf{Straight Lines}}
\date{\vspace{-5ex}} %Not to print date automatically
\maketitle
\end{center}
\setcounter{page}{1}
\section*{12$^{th}$ Maths - Chapter 10}
This is Problem-12 from Exercise 10.2
\begin{enumerate}
\item Find the direction cosines of the vector $\hat{i} +2\hat{j}+3\hat{k}$.

\solution The direction cosines are the cosines of the angles formed by the given vector with the respective axes, let $\vec{A}$ be the given vector
\begin{align}
	\vec{A} =\myvec{1\\2\\3}
\end{align}
The Directional vectors of $x,y$ and $z$ axes are given respectively 
\begin{align}
		\vec{e_1} =\myvec{1\\0\\0},\vec{e_2}=\myvec{0\\1\\0},\vec{e_3} =\myvec{0\\0\\1}
\end{align}
		The magnitudes for $\vec{A}$ and directional vectors $\vec{e_1},\vec{e_2},\vec{e_3}$ are
	\begin{align}
\norm{\vec{A}} =\sqrt{14},\norm{\vec{e_1}}=\sqrt{1},\norm{\vec{e_1}}=\sqrt{1},\norm{\vec{e_1}}=\sqrt{1}
	\end{align}
The direction cosines of $\vec{A}$ can be expressed as
\begin{align}
	\cos\theta_1 &=\frac{\vec{A}^\top\vec{e_1}}{\norm{\vec{A}}\norm{\vec{e_1}}}\\
	\cos\theta_2 &=\frac{\vec{A}^\top\vec{e_2}}{\norm{\vec{A}}\norm{\vec{e_2}}}\\
	\cos\theta_3 &=\frac{\vec{A}^\top\vec{e_3}}{\norm{\vec{A}}\norm{\vec{e_3}}}
\end{align}
		So the direction cosines of vector $\vec{A}$ are
\begin{align}
	\cos\theta_1 &=\frac{\myvec{1&2&3}\myvec{1\\0\\0}}{\sqrt{14}}=\frac{1}{\sqrt{14}}\\
	\cos\theta_2 &=\frac{\myvec{1&2&3}\myvec{0\\1\\0}}{\sqrt{14}}=\frac{2}{\sqrt{14}}\\
	\cos\theta_3 &=\frac{\myvec{1&2&3}\myvec{0\\0\\1}}{\sqrt{14}}=\frac{3}{\sqrt{14}}
\end{align}
\end{enumerate}
\end{document}
