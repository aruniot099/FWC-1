\documentclass[12pt]{article}
\usepackage{graphicx}
%\documentclass[journal,12pt,twocolumn]{IEEEtran}
\usepackage[none]{hyphenat}
\usepackage{graphicx}
\usepackage{listings}
\usepackage[english]{babel}
\usepackage{graphicx}
\usepackage{caption}
\usepackage[parfill]{parskip}
\usepackage{hyperref}
\usepackage{booktabs}
%\usepackage{setspace}\doublespacing\pagestyle{plain}
\def\inputGnumericTable{}
\usepackage{color}                                            %%
    \usepackage{array}                                            %%
    \usepackage{longtable}                                        %%
    \usepackage{calc}                                             %%
    \usepackage{multirow}                                         %%
    \usepackage{hhline}                                           %%
    \usepackage{ifthen}
\usepackage{array}
\usepackage{amsmath}   % for having text in math mode
\usepackage{parallel,enumitem}
\usepackage{listings}
\lstset{
language=tex,
frame=single,
breaklines=true
}
 
%Following 2 lines were added to remove the blank page at the beginning
\usepackage{atbegshi}% http://ctan.org/pkg/atbegshi
\AtBeginDocument{\AtBeginShipoutNext{\AtBeginShipoutDiscard}}
%
%New macro definitions
\newcommand{\mydet}[1]{\ensuremath{\begin{vmatrix}#1\end{vmatrix}}}
\providecommand{\brak}[1]{\ensuremath{\left(#1\right)}}
\providecommand{\norm}[1]{\left\lVert#1\right\rVert}
\newcommand{\solution}{\noindent \textbf{Solution: }}
\newcommand{\myvec}[1]{\ensuremath{\begin{pmatrix}#1\end{pmatrix}}}
\let\vec\mathbf
\begin{document}
\begin{center}
\enlargethispage{-4cm}
\title{\textbf{Straight Lines}}
\date{\vspace{-5ex}} %Not to print date automatically
\maketitle
\end{center}
\setcounter{page}{1}
\section*{12$^{th}$ Maths - Chapter 10}
This is Problem-12 from Exercise 10.2
\begin{enumerate}
\item Find the direction cosines of the vector $\hat{i} +2\hat{j}+3\hat{k}$.

\solution Let
\begin{align}
a=1, b=2, c=3, \vec{A}=\myvec{1\\2\\3}
\end{align}
The magnitude of $\vec{A}$ is given by
\begin{align}
\norm{\vec{A}}&={\vec{A}}^{\top}\vec{A}\\
\norm{\vec{A}} &=\sqrt{14}
\end{align}
The direction cosines are given by
\begin{align}
\frac{a}{\norm{\vec{A}}},\frac{b}{\norm{\vec{A}}},\frac{c}{\norm{\vec{A}}}\\
\implies\frac{1}{\sqrt{14}},\frac{2}{\sqrt{14}},\frac{3}{\sqrt{14}}
\end{align}



\end{enumerate}
\end{document}