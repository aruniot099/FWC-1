\documentclass[12pt]{article}
\usepackage{graphicx}
%\documentclass[journal,12pt,twocolumn]{IEEEtran}
\usepackage[none]{hyphenat}
\usepackage{graphicx}
\usepackage{listings}
\usepackage[english]{babel}
\usepackage{graphicx}
\usepackage{caption} 
\usepackage{hyperref}
\usepackage{booktabs}
\def\inputGnumericTable{}
\usepackage{color}                                            %%
    \usepackage{array}                                            %%
    \usepackage{longtable}                                        %%
    \usepackage{calc}                                             %%
    \usepackage{multirow}                                         %%
    \usepackage{hhline}                                           %%
    \usepackage{ifthen}
\usepackage{array}
\usepackage{amsmath}   % for having text in math mode
\usepackage{parallel,enumitem}
\usepackage{listings}
\lstset{
language=tex,
frame=single, 
breaklines=true
}
  
%Following 2 lines were added to remove the blank page at the beginning
\usepackage{atbegshi}% http://ctan.org/pkg/atbegshi
\AtBeginDocument{\AtBeginShipoutNext{\AtBeginShipoutDiscard}}
%
%New macro definitions
\newcommand{\mydet}[1]{\ensuremath{\begin{vmatrix}#1\end{vmatrix}}}
\providecommand{\brak}[1]{\ensuremath{\left(#1\right)}}
\providecommand{\norm}[1]{\left\lVert#1\right\rVert}
\newcommand{\solution}{\noindent \textbf{Solution: }}
\newcommand{\myvec}[1]{\ensuremath{\begin{pmatrix}#1\end{pmatrix}}}
\let\vec\mathbf
\begin{document}
\begin{center}
\title{\textbf{Coordinate Geometry}}
\date{\vspace{-5ex}} %Not to print date automatically
\maketitle
\end{center}
\setcounter{page}{1}
\section*{10$^{th}$ Maths - Chapter 7}
This is Problem-6 from Exercise 7.4
\begin{enumerate}
\item The vertices of $\triangle ABC$ are $\myvec{4 \\ 6}, \myvec{1\\5}, \myvec{7\\2}$. A line is drawn to intersect sides AB and AC at D and E respectively, such that $\dfrac{AD}{AB}=\dfrac{AE}{AC}=\dfrac{1}{4}$. Calculate the area of the $\triangle ADE$ and compare it with the area of $\triangle ABC$.\\
\solution The input parameters for this problem are available in Table \eqref{Table-1}
\begin{table}[ht!]\centering
\input{tables/table.tex}
\caption{}
\label{Table-1}	
\end{table}


Given,
\begin{align}
\frac{\vec{A}\vec{D}}{\vec{A}\vec{B}}=\frac{\vec{A}\vec{E}}{\vec{A}\vec{C}}=\frac{1}{4}\label{1}
\end{align}
From \eqref{1},
\begin{align}
\frac{\vec{A}\vec{D}}{\vec{A}\vec{B}} &=\frac{1}{4}\\
4 \vec{A}\vec{D} &=\vec{A}\vec{D} + \vec{B}\vec{D}\\
4 \vec{A}\vec{D}-\vec{A}\vec{D} &=\vec{B}\vec{D}\\
3 \vec{A}\vec{D} &=\vec{B}\vec{D}\\
\frac{\vec{A}\vec{D}}{\vec{B}\vec{D}} &=\frac{1}{3}
\end{align}
Point $\vec{D}$ divides $\vec{A}\vec{B}$ in the ratio of $n = 1:3$.\\
using Section formula,
\begin{align}
\vec{D} &=\frac{\vec{B}+n\vec{A}}{1+n}\label{7}
\end{align}
Substituting the values $\vec{A}$ and $\vec{B}$ in \eqref{7},
\begin{align}
\vec{D} &=\frac{{\myvec{1\\5}+n\myvec{4\\6}}}{1+n}\\
          &=\frac{1}{1+n}\brak{{\myvec{1\\5}+n\myvec{4\\6}}} \\
         &=\frac{1}{1+n}\myvec{4+n\\6+5n} \label{10}
\end{align}
Substituting the value of $n$ in \eqref{10},
\begin{align}
\vec{D} &=\frac{3}{4}\myvec{4+\cfrac{1}{3}\\6+\cfrac{5}{3}}\\
		&=\frac{3}{4}\myvec{\cfrac{13}{3}\\ \cfrac{23}{4}}\\
		&=\myvec{\cfrac{13}{4}\\ \cfrac{23}{4}}
\end{align}
%part-2

From \eqref{1},
\begin{align}
\frac{\vec{A}\vec{E}}{\vec{A}\vec{C}} &=\frac{1}{4}\\
4 \vec{A}\vec{E} &=\vec{A}\vec{E} + \vec{C}\vec{E}\\
4 \vec{A}\vec{E}-\vec{A}\vec{E} &=\vec{C}\vec{E}\\
3 \vec{A}\vec{E} &=\vec{C}\vec{E}\\
\frac{\vec{A}\vec{E}}{\vec{C}\vec{E}} &=\frac{1}{3}
\end{align}
Point $\vec{E}$ divides $\vec{A}\vec{C}$ in the ratio of $n = \frac{1}{3}$.\\
using Section formula,
\begin{align}
\vec{E} &=\frac{\vec{C}+n\vec{A}}{1+n}\label{19}
\end{align}
Substituting the values $\vec{A}$ and $\vec{C}$ in \eqref{19},
\begin{align}
\vec{E} &=\frac{{\myvec{7\\2}+n\myvec{4\\6}}}{1+n}\\
          &=\frac{1}{1+n}\brak{{\myvec{7\\2}+n\myvec{4\\6}}} \\
         &=\frac{1}{1+n}\myvec{4+7n\\6+2n} \label{22}
\end{align}
Substituting the value of $n$ in \eqref{22},
\begin{align}
\vec{E} &=\frac{3}{4}\myvec{4+\cfrac{7}{3}\\6+\cfrac{2}{3}}\\
		&=\frac{3}{4}\myvec{\cfrac{19}{3}\\ \cfrac{20}{3}}\\
		&=\myvec{\cfrac{19}{4}\\ \cfrac{20}{4}}
\end{align}
Now,\\
		The ar(ADE) can be expressed as
  \begin{align}
	  ar(ABD)=\frac{1}{2} \norm{\brak{\vec{A}-\vec{D}}  \times 
   \brak{\vec{A}- \vec{E}}} \label{eq:26} 
\end{align}
\begin{align}
	\vec{A}- \vec{D} &= \myvec{4\\6}-\myvec{\cfrac{13}{4}\\ \cfrac{23}{4}\\}=\myvec{\cfrac{3}{4}\\ \cfrac{1}{4}\\}\label{eq:27}\\
	  \vec{A}- \vec{E} &= \myvec{4\\6\\}-\myvec{\cfrac{19}{4}\\ \cfrac{20}{4}\\}=\myvec{\cfrac{-3}{4}\\1\\}\label{eq:28}
  \end{align}
Substituting the values of \eqref{eq:27} and \eqref{eq:28} in \eqref{eq:26},
\begin{align}
	ar(ADE)=\frac{1}{2}\mydet{\cfrac{3}{4} & \cfrac{-3}{4}\\ \cfrac{1}{4} & 1}  
	&=	\frac{15}{32}
\end{align}
Now,		
		The ar(ABC) can be expressed as
  \begin{align}
	  ar(ABC)=\frac{1}{2} \norm{\brak{\vec{A}-\vec{B}}  \times 
   \brak{\vec{B}- \vec{C}}} \label{eq:30} 
\end{align}
\begin{align}
	\vec{A}- \vec{B} &= \myvec{4\\6\\}-\myvec{1\\5\\}=\myvec{3\\1\\}\label{eq:31}\\
	  \vec{B}-\vec{C} &= \myvec{1\\5\\}-\myvec{7\\2\\}=\myvec{-6\\3\\}\label{eq:32}
  \end{align}
Substituting the values of \eqref{eq:31} and \eqref{eq:32} in \eqref{eq:30},
\begin{align}
	ar(ABC)=\frac{1}{2}\mydet{3 & -6\\1 & 3}  
	&=	\frac{15}{2}
\end{align}
		Thus,
\begin{align}
	\frac{ar\brak{ADE}}{ar\brak{ABC}}&=\frac{\cfrac{15}{32}}{\cfrac{15}{2}}\\
	&=\frac{15}{32}\times \frac{2}{15}=\frac{1}{16}
\end{align}

\begin{figure}[!h]
 \begin{center}
 \includegraphics[width=\columnwidth]{figs/fig.png}
 \end{center}
\caption{}
\label{fig:Fig1}
\end{figure}
\end{enumerate}
\end{document}