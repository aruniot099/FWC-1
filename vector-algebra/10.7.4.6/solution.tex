\documentclass[12pt]{article}
\usepackage{graphicx}
%\documentclass[journal,12pt,twocolumn]{IEEEtran}
\usepackage[none]{hyphenat}
\usepackage{graphicx}
\usepackage{listings}
\usepackage[english]{babel}
\usepackage{graphicx}
\usepackage{caption}
\usepackage[parfill]{parskip}
\usepackage{hyperref}
\usepackage{booktabs}
%\usepackage{setspace}\doublespacing\pagestyle{plain}
\def\inputGnumericTable{}
\usepackage{color}                                            %%
    \usepackage{array}                                            %%
    \usepackage{longtable}                                        %%
    \usepackage{calc}                                             %%
    \usepackage{multirow}                                         %%
    \usepackage{hhline}                                           %%
    \usepackage{ifthen}
\usepackage{array}
\usepackage{amsmath}   % for having text in math mode
\usepackage{parallel,enumitem}
\usepackage{listings}
\lstset{
language=tex,
frame=single, 
breaklines=true
}
  
%Following 2 lines were added to remove the blank page at the beginning
\usepackage{atbegshi}% http://ctan.org/pkg/atbegshi
\AtBeginDocument{\AtBeginShipoutNext{\AtBeginShipoutDiscard}}
%
%New macro definitions
\newcommand{\mydet}[1]{\ensuremath{\begin{vmatrix}#1\end{vmatrix}}}
\providecommand{\brak}[1]{\ensuremath{\left(#1\right)}}
\providecommand{\norm}[1]{\left\lVert#1\right\rVert}
\newcommand{\solution}{\noindent \textbf{Solution: }}
\newcommand{\myvec}[1]{\ensuremath{\begin{pmatrix}#1\end{pmatrix}}}
\let\vec\mathbf
\begin{document}
\begin{center}
\title{\textbf{Coordinate Geometry}}
\date{\vspace{-5ex}} %Not to print date automatically
\maketitle
\end{center}
\setcounter{page}{1}
\section*{10$^{th}$ Maths - Chapter 7}
This is Problem-6 from Exercise 7.4
\begin{enumerate}
\item The vertices of $\triangle ABC$ are $\myvec{4 \\ 6}, \myvec{1\\5}, \myvec{7\\2}$. A line is drawn to intersect sides AB and AC at D and E respectively, such that $\dfrac{AD}{AB}=\dfrac{AE}{AC}=\dfrac{1}{4}$. Calculate the area of the $\triangle ADE$ and compare it with the area of $\triangle ABC$.\\
\solution The input parameters for this problem are available in Table \eqref{Table-1}
\begin{table}[ht!]\centering
\input{tables/table.tex}
\caption{}
\label{Table-1}	
\end{table}

Given,
\begin{align}
\frac{AD}{AB}=\frac{AE}{AC}=\frac{1}{4}\label{1}
\end{align}
From \eqref{1},
\begin{align}
\frac{AD}{AB} &=\frac{1}{4}\\
\frac{AD}{BD} &=\frac{1}{3}
\end{align}
Point $\vec{D}$ divides $\vec{A}\vec{B}$ in the ratio of $n = 1:3$.
	Using Section formula,
\begin{align}
\vec{D} &=\frac{\vec{A}+n\vec{B}}{1+n}\label{4}
\end{align}
Substituting the values $\vec{A},\vec{B}$ and $n$ in \eqref{4},
\begin{align}
\vec{D} &=\frac{{\myvec{4\\6}+\frac{1}{3}\myvec{1\\5}}}{1+\frac{1}{3}}\\
	&=\frac{1}{1+\frac{1}{3}}\brak{{\myvec{4\\6}+\frac{1}{3}\myvec{1\\5}}} \\
	&=\frac{3}{4}\myvec{4+\frac{1}{3}\\[2pt]6+\frac{5}{3}}\\
	&=\myvec{\frac{13}{4}\\[2pt] \frac{23}{4}}
\end{align}
%part-2

From \eqref{1},
\begin{align}
\frac{AE}{AC} &=\frac{1}{4}\\
\frac{AE}{CE} &=\frac{1}{3}
\end{align}
Point $\vec{E}$ divides $\vec{A}\vec{C}$ in the ratio of $n = \frac{1}{3}$.
Using Section formula,
\begin{align}
\vec{E} &=\frac{\vec{A}+n\vec{C}}{1+n}\label{11}
\end{align}
Substituting the values $\vec{A},\vec{C}$ and $n$ in \eqref{11},
\begin{align}
\vec{E} &=\frac{{\myvec{4\\6}+\frac{1}{3}\myvec{7\\2}}}{1+\frac{1}{3}}\\
	&=\frac{1}{1+\frac{1}{3}}\brak{{\myvec{4\\6}+\frac{1}{3}\myvec{7\\2}}} \\
	&=\frac{3}{4}\myvec{4+\frac{7}{3}\\[2pt]6+\frac{2}{3}}\\
	&=\myvec{\frac{19}{4}\\[2pt] \frac{20}{4}}
\end{align}
The ar(ADE) can be expressed as
\begin{align}
	ar(ABD) &=\frac{1}{2} \norm{\brak{\vec{A}-\vec{D}}  \times 
   \brak{\vec{A}- \vec{E}}} \label{eq:16} \\
	\vec{A}- \vec{D} &= \myvec{4\\6}-\myvec{\frac{13}{4}\\[2pt] \frac{23}{4}}=\myvec{\frac{3}{4}\\[2pt] \frac{1}{4}}\label{eq:17}\\
	  \vec{A}- \vec{E} &= \myvec{4\\6}-\myvec{\frac{19}{4}\\[2pt] \frac{20}{4}}=\myvec{\frac{-3}{4}\\[2pt]1}\label{eq:18}
  \end{align}
Substituting the values of \eqref{eq:17} and \eqref{eq:18} in \eqref{eq:16},
\begin{align}
	ar(ADE)=\frac{1}{2}\mydet{\frac{3}{4} & \frac{-3}{4}\\[2pt] \frac{1}{4} & 1}  
	&=	\frac{15}{32}
\end{align}
The ar(ABC) can be expressed as
  \begin{align}
	  ar(ABC) &=\frac{1}{2} \norm{\brak{\vec{A}-\vec{B}}  \times 
   \brak{\vec{B}- \vec{C}}} \label{eq:20} \\
	\vec{A}- \vec{B} &= \myvec{4\\6}-\myvec{1\\5}=\myvec{3\\1}\label{eq:21}\\
	  \vec{B}-\vec{C} &= \myvec{1\\5}-\myvec{7\\2}=\myvec{-6\\3}\label{eq:22}
  \end{align}
Substituting the values of \eqref{eq:21} and \eqref{eq:22} in \eqref{eq:20},
\begin{align}
	ar(ABC)=\frac{1}{2}\mydet{3 & -6\\1 & 3}  
	&=	\frac{15}{2}
\end{align}
		Comparing ar(ADE) with ar(ABC),
\begin{align}
	\frac{ar\brak{ADE}}{ar\brak{ABC}}&=\frac{\frac{15}{32}}{\frac{15}{2}}\\
	&=\frac{15}{32}\times \frac{2}{15}=\frac{1}{16}
\end{align}

\begin{figure}[!h]
 \begin{center}
 \includegraphics[width=\columnwidth]{figs/fig.png}
 \end{center}
\caption{}
\label{fig:Fig1}
\end{figure}
\end{enumerate}
\end{document}
