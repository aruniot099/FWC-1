\documentclass[12pt]{article}
\usepackage{graphicx}
\usepackage{amsmath}   
\usepackage{commath}
\usepackage{gensymb}


\begin{document}
\title{\textbf{Area of a Traingle}}
\maketitle
\begin{center}
\end{center}

\begin{enumerate}
\section*{12$^{th}$ Maths - Chapter 10}

\item Find $\abs{\overrightarrow{a}\times\overrightarrow{b}},\text{ if }\overrightarrow{a}=\hat{i}-7\hat{j}+7\hat{k}\text{ and } \overrightarrow{b}=3\hat{i}-2\hat{j}+2\hat{k}$.
\item Find a unit vector perpendicular to each of the vector $\overrightarrow{a}+\overrightarrow{b}\text{ and }\overrightarrow{a}-\overrightarrow{b},\text{ where } \overrightarrow{a}=3\hat{i}+2\hat{j}+2\hat{k}\text{ and } \overrightarrow{b}=\hat{i}+2\hat{j}-2\hat{k}$. 
\item If a unit vector $\overrightarrow{a}$ makes angles $\dfrac{\pi}{3}\text{ with }\hat{i}, \dfrac{\pi}{4}\text{ with }\hat{j}$ and an acute angle $\theta \text{ with }\hat{k},\text{ then find } \theta$ and hence, the components of $\overrightarrow{a}$.
\item Show that $$(\overrightarrow{a}-\overrightarrow{b})\times (\overrightarrow{a}+\overrightarrow{b})=2(\overrightarrow{a}\times \overrightarrow{b})$$
\item Find $\lambda$ and $\mu$ if $(2\hat{i}+6\hat{j}+27\hat{k})\times(\hat{i}+\lambda \hat{j} + \mu \hat{k})=\overrightarrow{0}$.
\item Given that $\overrightarrow{a} \cdot \overrightarrow{b} = 0$ and $\overrightarrow{a} \times \overrightarrow{b} = \overrightarrow{0}$. What can you conclude about the vectors $\overrightarrow{a} \text{ and }\overrightarrow{b}$?
\item Let the vectors be given as $\overrightarrow{a},\overrightarrow{b},\overrightarrow{c}\text{ be given as }\ a_1 \hat{i}+\ a_2 \hat{j}+\ a_3 \hat{k},\ b_1 \hat{i}+\ b_2 \hat{j}+\ b_3 \hat{k},\ c_1 \hat{i}+\ c_2 \hat{j}+\ c_3 \hat{k}$. Then show that $\overrightarrow{a} \times (\overrightarrow{b} + \overrightarrow{c}) = \overrightarrow{a} \times \overrightarrow{b}+\overrightarrow{a} \times \overrightarrow{c}$.
\item If either $\overrightarrow{a} = \overrightarrow{0}$ or $\overrightarrow{b} = \overrightarrow{0}$, then $\overrightarrow{a} \times \overrightarrow{b} = \overrightarrow{0}$. Is the converse true? Justify your answer with an example.
\item Find the area of the triangle with vertices $A(1, 1, 2)$, $B(2, 3, 5)$, and $C(1, 5, 5)$
\item Find the area of the parallelogram whose adjacent sides are determined by the vectors $\overrightarrow{a}=\hat{i}-\hat{j}+3\hat{k}$ and $\overrightarrow{b}=2\hat{i}-7\hat{j}+\hat{k}$.
\item Let the vectors $\overrightarrow{a}$ and $\overrightarrow{b}$ be such that $|\overrightarrow{a}| = 3$ and $|\overrightarrow{b}| = \dfrac{\sqrt{2}}{3}$, then $\overrightarrow{a} \times \overrightarrow{b}$ is a unit vector, if the angle between $\overrightarrow{a}$ and $\overrightarrow{b}$ is
\begin{enumerate}
\item $\dfrac{\pi}{6}$
\item $\dfrac{\pi}{4}$
\item $\dfrac{\pi}{3}$
\item $\dfrac{\pi}{2}$
\end{enumerate}
\item Area of a rectangle having vertices A, B, C and D with position vectors $ -\hat{i}+ \dfrac{1}{2} \hat{j}+4\hat{k},\hat{i}+ \dfrac{1}{2} \hat{j}+4\hat{k},\hat{i}-\dfrac{1}{2} \hat{j}+4\hat{k}\text{ and }-\hat{i}- \dfrac{1}{2} \hat{j}+4\hat{k}$, respectively is
\begin{enumerate}
\item $\dfrac{1}{2}$
\item 1
\item 2
\item 4
\end{enumerate}
\end{enumerate}
\end{document}
